\documentclass[11pt]{article}

    \usepackage[breakable]{tcolorbox}
    \usepackage{parskip} % Stop auto-indenting (to mimic markdown behaviour)
    
    \usepackage{iftex}
    \ifPDFTeX
    	\usepackage[T1]{fontenc}
    	\usepackage{mathpazo}
    \else
    	\usepackage{fontspec}
    \fi

    % Basic figure setup, for now with no caption control since it's done
    % automatically by Pandoc (which extracts ![](path) syntax from Markdown).
    \usepackage{graphicx}
    % Maintain compatibility with old templates. Remove in nbconvert 6.0
    \let\Oldincludegraphics\includegraphics
    % Ensure that by default, figures have no caption (until we provide a
    % proper Figure object with a Caption API and a way to capture that
    % in the conversion process - todo).
    \usepackage{caption}
    \DeclareCaptionFormat{nocaption}{}
    \captionsetup{format=nocaption,aboveskip=0pt,belowskip=0pt}

    \usepackage[Export]{adjustbox} % Used to constrain images to a maximum size
    \adjustboxset{max size={0.9\linewidth}{0.9\paperheight}}
    \usepackage{float}
    \floatplacement{figure}{H} % forces figures to be placed at the correct location
    \usepackage{xcolor} % Allow colors to be defined
    \usepackage{enumerate} % Needed for markdown enumerations to work
    \usepackage{geometry} % Used to adjust the document margins
    \usepackage{amsmath} % Equations
    \usepackage{amssymb} % Equations
    \usepackage{textcomp} % defines textquotesingle
    % Hack from http://tex.stackexchange.com/a/47451/13684:
    \AtBeginDocument{%
        \def\PYZsq{\textquotesingle}% Upright quotes in Pygmentized code
    }
    \usepackage{upquote} % Upright quotes for verbatim code
    \usepackage{eurosym} % defines \euro
    \usepackage[mathletters]{ucs} % Extended unicode (utf-8) support
    \usepackage{fancyvrb} % verbatim replacement that allows latex
    \usepackage{grffile} % extends the file name processing of package graphics 
                         % to support a larger range
    \makeatletter % fix for grffile with XeLaTeX
    \def\Gread@@xetex#1{%
      \IfFileExists{"\Gin@base".bb}%
      {\Gread@eps{\Gin@base.bb}}%
      {\Gread@@xetex@aux#1}%
    }
    \makeatother

    % The hyperref package gives us a pdf with properly built
    % internal navigation ('pdf bookmarks' for the table of contents,
    % internal cross-reference links, web links for URLs, etc.)
    \usepackage{hyperref}
    % The default LaTeX title has an obnoxious amount of whitespace. By default,
    % titling removes some of it. It also provides customization options.
    \usepackage{titling}
    \usepackage{longtable} % longtable support required by pandoc >1.10
    \usepackage{booktabs}  % table support for pandoc > 1.12.2
    \usepackage[inline]{enumitem} % IRkernel/repr support (it uses the enumerate* environment)
    \usepackage[normalem]{ulem} % ulem is needed to support strikethroughs (\sout)
                                % normalem makes italics be italics, not underlines
    \usepackage{mathrsfs}
    

    
    % Colors for the hyperref package
    \definecolor{urlcolor}{rgb}{0,.145,.698}
    \definecolor{linkcolor}{rgb}{.71,0.21,0.01}
    \definecolor{citecolor}{rgb}{.12,.54,.11}

    % ANSI colors
    \definecolor{ansi-black}{HTML}{3E424D}
    \definecolor{ansi-black-intense}{HTML}{282C36}
    \definecolor{ansi-red}{HTML}{E75C58}
    \definecolor{ansi-red-intense}{HTML}{B22B31}
    \definecolor{ansi-green}{HTML}{00A250}
    \definecolor{ansi-green-intense}{HTML}{007427}
    \definecolor{ansi-yellow}{HTML}{DDB62B}
    \definecolor{ansi-yellow-intense}{HTML}{B27D12}
    \definecolor{ansi-blue}{HTML}{208FFB}
    \definecolor{ansi-blue-intense}{HTML}{0065CA}
    \definecolor{ansi-magenta}{HTML}{D160C4}
    \definecolor{ansi-magenta-intense}{HTML}{A03196}
    \definecolor{ansi-cyan}{HTML}{60C6C8}
    \definecolor{ansi-cyan-intense}{HTML}{258F8F}
    \definecolor{ansi-white}{HTML}{C5C1B4}
    \definecolor{ansi-white-intense}{HTML}{A1A6B2}
    \definecolor{ansi-default-inverse-fg}{HTML}{FFFFFF}
    \definecolor{ansi-default-inverse-bg}{HTML}{000000}

    % commands and environments needed by pandoc snippets
    % extracted from the output of `pandoc -s`
    \providecommand{\tightlist}{%
      \setlength{\itemsep}{0pt}\setlength{\parskip}{0pt}}
    \DefineVerbatimEnvironment{Highlighting}{Verbatim}{commandchars=\\\{\}}
    % Add ',fontsize=\small' for more characters per line
    \newenvironment{Shaded}{}{}
    \newcommand{\KeywordTok}[1]{\textcolor[rgb]{0.00,0.44,0.13}{\textbf{{#1}}}}
    \newcommand{\DataTypeTok}[1]{\textcolor[rgb]{0.56,0.13,0.00}{{#1}}}
    \newcommand{\DecValTok}[1]{\textcolor[rgb]{0.25,0.63,0.44}{{#1}}}
    \newcommand{\BaseNTok}[1]{\textcolor[rgb]{0.25,0.63,0.44}{{#1}}}
    \newcommand{\FloatTok}[1]{\textcolor[rgb]{0.25,0.63,0.44}{{#1}}}
    \newcommand{\CharTok}[1]{\textcolor[rgb]{0.25,0.44,0.63}{{#1}}}
    \newcommand{\StringTok}[1]{\textcolor[rgb]{0.25,0.44,0.63}{{#1}}}
    \newcommand{\CommentTok}[1]{\textcolor[rgb]{0.38,0.63,0.69}{\textit{{#1}}}}
    \newcommand{\OtherTok}[1]{\textcolor[rgb]{0.00,0.44,0.13}{{#1}}}
    \newcommand{\AlertTok}[1]{\textcolor[rgb]{1.00,0.00,0.00}{\textbf{{#1}}}}
    \newcommand{\FunctionTok}[1]{\textcolor[rgb]{0.02,0.16,0.49}{{#1}}}
    \newcommand{\RegionMarkerTok}[1]{{#1}}
    \newcommand{\ErrorTok}[1]{\textcolor[rgb]{1.00,0.00,0.00}{\textbf{{#1}}}}
    \newcommand{\NormalTok}[1]{{#1}}
    
    % Additional commands for more recent versions of Pandoc
    \newcommand{\ConstantTok}[1]{\textcolor[rgb]{0.53,0.00,0.00}{{#1}}}
    \newcommand{\SpecialCharTok}[1]{\textcolor[rgb]{0.25,0.44,0.63}{{#1}}}
    \newcommand{\VerbatimStringTok}[1]{\textcolor[rgb]{0.25,0.44,0.63}{{#1}}}
    \newcommand{\SpecialStringTok}[1]{\textcolor[rgb]{0.73,0.40,0.53}{{#1}}}
    \newcommand{\ImportTok}[1]{{#1}}
    \newcommand{\DocumentationTok}[1]{\textcolor[rgb]{0.73,0.13,0.13}{\textit{{#1}}}}
    \newcommand{\AnnotationTok}[1]{\textcolor[rgb]{0.38,0.63,0.69}{\textbf{\textit{{#1}}}}}
    \newcommand{\CommentVarTok}[1]{\textcolor[rgb]{0.38,0.63,0.69}{\textbf{\textit{{#1}}}}}
    \newcommand{\VariableTok}[1]{\textcolor[rgb]{0.10,0.09,0.49}{{#1}}}
    \newcommand{\ControlFlowTok}[1]{\textcolor[rgb]{0.00,0.44,0.13}{\textbf{{#1}}}}
    \newcommand{\OperatorTok}[1]{\textcolor[rgb]{0.40,0.40,0.40}{{#1}}}
    \newcommand{\BuiltInTok}[1]{{#1}}
    \newcommand{\ExtensionTok}[1]{{#1}}
    \newcommand{\PreprocessorTok}[1]{\textcolor[rgb]{0.74,0.48,0.00}{{#1}}}
    \newcommand{\AttributeTok}[1]{\textcolor[rgb]{0.49,0.56,0.16}{{#1}}}
    \newcommand{\InformationTok}[1]{\textcolor[rgb]{0.38,0.63,0.69}{\textbf{\textit{{#1}}}}}
    \newcommand{\WarningTok}[1]{\textcolor[rgb]{0.38,0.63,0.69}{\textbf{\textit{{#1}}}}}
    
    
    % Define a nice break command that doesn't care if a line doesn't already
    % exist.
    \def\br{\hspace*{\fill} \\* }
    % Math Jax compatibility definitions
    \def\gt{>}
    \def\lt{<}
    \let\Oldtex\TeX
    \let\Oldlatex\LaTeX
    \renewcommand{\TeX}{\textrm{\Oldtex}}
    \renewcommand{\LaTeX}{\textrm{\Oldlatex}}
    % Document parameters
    % Document title
    \title{DP0701EN-3-3-1-Clustering-k-means-py-v1.0}
    
    
    
    
    
% Pygments definitions
\makeatletter
\def\PY@reset{\let\PY@it=\relax \let\PY@bf=\relax%
    \let\PY@ul=\relax \let\PY@tc=\relax%
    \let\PY@bc=\relax \let\PY@ff=\relax}
\def\PY@tok#1{\csname PY@tok@#1\endcsname}
\def\PY@toks#1+{\ifx\relax#1\empty\else%
    \PY@tok{#1}\expandafter\PY@toks\fi}
\def\PY@do#1{\PY@bc{\PY@tc{\PY@ul{%
    \PY@it{\PY@bf{\PY@ff{#1}}}}}}}
\def\PY#1#2{\PY@reset\PY@toks#1+\relax+\PY@do{#2}}

\expandafter\def\csname PY@tok@w\endcsname{\def\PY@tc##1{\textcolor[rgb]{0.73,0.73,0.73}{##1}}}
\expandafter\def\csname PY@tok@c\endcsname{\let\PY@it=\textit\def\PY@tc##1{\textcolor[rgb]{0.25,0.50,0.50}{##1}}}
\expandafter\def\csname PY@tok@cp\endcsname{\def\PY@tc##1{\textcolor[rgb]{0.74,0.48,0.00}{##1}}}
\expandafter\def\csname PY@tok@k\endcsname{\let\PY@bf=\textbf\def\PY@tc##1{\textcolor[rgb]{0.00,0.50,0.00}{##1}}}
\expandafter\def\csname PY@tok@kp\endcsname{\def\PY@tc##1{\textcolor[rgb]{0.00,0.50,0.00}{##1}}}
\expandafter\def\csname PY@tok@kt\endcsname{\def\PY@tc##1{\textcolor[rgb]{0.69,0.00,0.25}{##1}}}
\expandafter\def\csname PY@tok@o\endcsname{\def\PY@tc##1{\textcolor[rgb]{0.40,0.40,0.40}{##1}}}
\expandafter\def\csname PY@tok@ow\endcsname{\let\PY@bf=\textbf\def\PY@tc##1{\textcolor[rgb]{0.67,0.13,1.00}{##1}}}
\expandafter\def\csname PY@tok@nb\endcsname{\def\PY@tc##1{\textcolor[rgb]{0.00,0.50,0.00}{##1}}}
\expandafter\def\csname PY@tok@nf\endcsname{\def\PY@tc##1{\textcolor[rgb]{0.00,0.00,1.00}{##1}}}
\expandafter\def\csname PY@tok@nc\endcsname{\let\PY@bf=\textbf\def\PY@tc##1{\textcolor[rgb]{0.00,0.00,1.00}{##1}}}
\expandafter\def\csname PY@tok@nn\endcsname{\let\PY@bf=\textbf\def\PY@tc##1{\textcolor[rgb]{0.00,0.00,1.00}{##1}}}
\expandafter\def\csname PY@tok@ne\endcsname{\let\PY@bf=\textbf\def\PY@tc##1{\textcolor[rgb]{0.82,0.25,0.23}{##1}}}
\expandafter\def\csname PY@tok@nv\endcsname{\def\PY@tc##1{\textcolor[rgb]{0.10,0.09,0.49}{##1}}}
\expandafter\def\csname PY@tok@no\endcsname{\def\PY@tc##1{\textcolor[rgb]{0.53,0.00,0.00}{##1}}}
\expandafter\def\csname PY@tok@nl\endcsname{\def\PY@tc##1{\textcolor[rgb]{0.63,0.63,0.00}{##1}}}
\expandafter\def\csname PY@tok@ni\endcsname{\let\PY@bf=\textbf\def\PY@tc##1{\textcolor[rgb]{0.60,0.60,0.60}{##1}}}
\expandafter\def\csname PY@tok@na\endcsname{\def\PY@tc##1{\textcolor[rgb]{0.49,0.56,0.16}{##1}}}
\expandafter\def\csname PY@tok@nt\endcsname{\let\PY@bf=\textbf\def\PY@tc##1{\textcolor[rgb]{0.00,0.50,0.00}{##1}}}
\expandafter\def\csname PY@tok@nd\endcsname{\def\PY@tc##1{\textcolor[rgb]{0.67,0.13,1.00}{##1}}}
\expandafter\def\csname PY@tok@s\endcsname{\def\PY@tc##1{\textcolor[rgb]{0.73,0.13,0.13}{##1}}}
\expandafter\def\csname PY@tok@sd\endcsname{\let\PY@it=\textit\def\PY@tc##1{\textcolor[rgb]{0.73,0.13,0.13}{##1}}}
\expandafter\def\csname PY@tok@si\endcsname{\let\PY@bf=\textbf\def\PY@tc##1{\textcolor[rgb]{0.73,0.40,0.53}{##1}}}
\expandafter\def\csname PY@tok@se\endcsname{\let\PY@bf=\textbf\def\PY@tc##1{\textcolor[rgb]{0.73,0.40,0.13}{##1}}}
\expandafter\def\csname PY@tok@sr\endcsname{\def\PY@tc##1{\textcolor[rgb]{0.73,0.40,0.53}{##1}}}
\expandafter\def\csname PY@tok@ss\endcsname{\def\PY@tc##1{\textcolor[rgb]{0.10,0.09,0.49}{##1}}}
\expandafter\def\csname PY@tok@sx\endcsname{\def\PY@tc##1{\textcolor[rgb]{0.00,0.50,0.00}{##1}}}
\expandafter\def\csname PY@tok@m\endcsname{\def\PY@tc##1{\textcolor[rgb]{0.40,0.40,0.40}{##1}}}
\expandafter\def\csname PY@tok@gh\endcsname{\let\PY@bf=\textbf\def\PY@tc##1{\textcolor[rgb]{0.00,0.00,0.50}{##1}}}
\expandafter\def\csname PY@tok@gu\endcsname{\let\PY@bf=\textbf\def\PY@tc##1{\textcolor[rgb]{0.50,0.00,0.50}{##1}}}
\expandafter\def\csname PY@tok@gd\endcsname{\def\PY@tc##1{\textcolor[rgb]{0.63,0.00,0.00}{##1}}}
\expandafter\def\csname PY@tok@gi\endcsname{\def\PY@tc##1{\textcolor[rgb]{0.00,0.63,0.00}{##1}}}
\expandafter\def\csname PY@tok@gr\endcsname{\def\PY@tc##1{\textcolor[rgb]{1.00,0.00,0.00}{##1}}}
\expandafter\def\csname PY@tok@ge\endcsname{\let\PY@it=\textit}
\expandafter\def\csname PY@tok@gs\endcsname{\let\PY@bf=\textbf}
\expandafter\def\csname PY@tok@gp\endcsname{\let\PY@bf=\textbf\def\PY@tc##1{\textcolor[rgb]{0.00,0.00,0.50}{##1}}}
\expandafter\def\csname PY@tok@go\endcsname{\def\PY@tc##1{\textcolor[rgb]{0.53,0.53,0.53}{##1}}}
\expandafter\def\csname PY@tok@gt\endcsname{\def\PY@tc##1{\textcolor[rgb]{0.00,0.27,0.87}{##1}}}
\expandafter\def\csname PY@tok@err\endcsname{\def\PY@bc##1{\setlength{\fboxsep}{0pt}\fcolorbox[rgb]{1.00,0.00,0.00}{1,1,1}{\strut ##1}}}
\expandafter\def\csname PY@tok@kc\endcsname{\let\PY@bf=\textbf\def\PY@tc##1{\textcolor[rgb]{0.00,0.50,0.00}{##1}}}
\expandafter\def\csname PY@tok@kd\endcsname{\let\PY@bf=\textbf\def\PY@tc##1{\textcolor[rgb]{0.00,0.50,0.00}{##1}}}
\expandafter\def\csname PY@tok@kn\endcsname{\let\PY@bf=\textbf\def\PY@tc##1{\textcolor[rgb]{0.00,0.50,0.00}{##1}}}
\expandafter\def\csname PY@tok@kr\endcsname{\let\PY@bf=\textbf\def\PY@tc##1{\textcolor[rgb]{0.00,0.50,0.00}{##1}}}
\expandafter\def\csname PY@tok@bp\endcsname{\def\PY@tc##1{\textcolor[rgb]{0.00,0.50,0.00}{##1}}}
\expandafter\def\csname PY@tok@fm\endcsname{\def\PY@tc##1{\textcolor[rgb]{0.00,0.00,1.00}{##1}}}
\expandafter\def\csname PY@tok@vc\endcsname{\def\PY@tc##1{\textcolor[rgb]{0.10,0.09,0.49}{##1}}}
\expandafter\def\csname PY@tok@vg\endcsname{\def\PY@tc##1{\textcolor[rgb]{0.10,0.09,0.49}{##1}}}
\expandafter\def\csname PY@tok@vi\endcsname{\def\PY@tc##1{\textcolor[rgb]{0.10,0.09,0.49}{##1}}}
\expandafter\def\csname PY@tok@vm\endcsname{\def\PY@tc##1{\textcolor[rgb]{0.10,0.09,0.49}{##1}}}
\expandafter\def\csname PY@tok@sa\endcsname{\def\PY@tc##1{\textcolor[rgb]{0.73,0.13,0.13}{##1}}}
\expandafter\def\csname PY@tok@sb\endcsname{\def\PY@tc##1{\textcolor[rgb]{0.73,0.13,0.13}{##1}}}
\expandafter\def\csname PY@tok@sc\endcsname{\def\PY@tc##1{\textcolor[rgb]{0.73,0.13,0.13}{##1}}}
\expandafter\def\csname PY@tok@dl\endcsname{\def\PY@tc##1{\textcolor[rgb]{0.73,0.13,0.13}{##1}}}
\expandafter\def\csname PY@tok@s2\endcsname{\def\PY@tc##1{\textcolor[rgb]{0.73,0.13,0.13}{##1}}}
\expandafter\def\csname PY@tok@sh\endcsname{\def\PY@tc##1{\textcolor[rgb]{0.73,0.13,0.13}{##1}}}
\expandafter\def\csname PY@tok@s1\endcsname{\def\PY@tc##1{\textcolor[rgb]{0.73,0.13,0.13}{##1}}}
\expandafter\def\csname PY@tok@mb\endcsname{\def\PY@tc##1{\textcolor[rgb]{0.40,0.40,0.40}{##1}}}
\expandafter\def\csname PY@tok@mf\endcsname{\def\PY@tc##1{\textcolor[rgb]{0.40,0.40,0.40}{##1}}}
\expandafter\def\csname PY@tok@mh\endcsname{\def\PY@tc##1{\textcolor[rgb]{0.40,0.40,0.40}{##1}}}
\expandafter\def\csname PY@tok@mi\endcsname{\def\PY@tc##1{\textcolor[rgb]{0.40,0.40,0.40}{##1}}}
\expandafter\def\csname PY@tok@il\endcsname{\def\PY@tc##1{\textcolor[rgb]{0.40,0.40,0.40}{##1}}}
\expandafter\def\csname PY@tok@mo\endcsname{\def\PY@tc##1{\textcolor[rgb]{0.40,0.40,0.40}{##1}}}
\expandafter\def\csname PY@tok@ch\endcsname{\let\PY@it=\textit\def\PY@tc##1{\textcolor[rgb]{0.25,0.50,0.50}{##1}}}
\expandafter\def\csname PY@tok@cm\endcsname{\let\PY@it=\textit\def\PY@tc##1{\textcolor[rgb]{0.25,0.50,0.50}{##1}}}
\expandafter\def\csname PY@tok@cpf\endcsname{\let\PY@it=\textit\def\PY@tc##1{\textcolor[rgb]{0.25,0.50,0.50}{##1}}}
\expandafter\def\csname PY@tok@c1\endcsname{\let\PY@it=\textit\def\PY@tc##1{\textcolor[rgb]{0.25,0.50,0.50}{##1}}}
\expandafter\def\csname PY@tok@cs\endcsname{\let\PY@it=\textit\def\PY@tc##1{\textcolor[rgb]{0.25,0.50,0.50}{##1}}}

\def\PYZbs{\char`\\}
\def\PYZus{\char`\_}
\def\PYZob{\char`\{}
\def\PYZcb{\char`\}}
\def\PYZca{\char`\^}
\def\PYZam{\char`\&}
\def\PYZlt{\char`\<}
\def\PYZgt{\char`\>}
\def\PYZsh{\char`\#}
\def\PYZpc{\char`\%}
\def\PYZdl{\char`\$}
\def\PYZhy{\char`\-}
\def\PYZsq{\char`\'}
\def\PYZdq{\char`\"}
\def\PYZti{\char`\~}
% for compatibility with earlier versions
\def\PYZat{@}
\def\PYZlb{[}
\def\PYZrb{]}
\makeatother


    % For linebreaks inside Verbatim environment from package fancyvrb. 
    \makeatletter
        \newbox\Wrappedcontinuationbox 
        \newbox\Wrappedvisiblespacebox 
        \newcommand*\Wrappedvisiblespace {\textcolor{red}{\textvisiblespace}} 
        \newcommand*\Wrappedcontinuationsymbol {\textcolor{red}{\llap{\tiny$\m@th\hookrightarrow$}}} 
        \newcommand*\Wrappedcontinuationindent {3ex } 
        \newcommand*\Wrappedafterbreak {\kern\Wrappedcontinuationindent\copy\Wrappedcontinuationbox} 
        % Take advantage of the already applied Pygments mark-up to insert 
        % potential linebreaks for TeX processing. 
        %        {, <, #, %, $, ' and ": go to next line. 
        %        _, }, ^, &, >, - and ~: stay at end of broken line. 
        % Use of \textquotesingle for straight quote. 
        \newcommand*\Wrappedbreaksatspecials {% 
            \def\PYGZus{\discretionary{\char`\_}{\Wrappedafterbreak}{\char`\_}}% 
            \def\PYGZob{\discretionary{}{\Wrappedafterbreak\char`\{}{\char`\{}}% 
            \def\PYGZcb{\discretionary{\char`\}}{\Wrappedafterbreak}{\char`\}}}% 
            \def\PYGZca{\discretionary{\char`\^}{\Wrappedafterbreak}{\char`\^}}% 
            \def\PYGZam{\discretionary{\char`\&}{\Wrappedafterbreak}{\char`\&}}% 
            \def\PYGZlt{\discretionary{}{\Wrappedafterbreak\char`\<}{\char`\<}}% 
            \def\PYGZgt{\discretionary{\char`\>}{\Wrappedafterbreak}{\char`\>}}% 
            \def\PYGZsh{\discretionary{}{\Wrappedafterbreak\char`\#}{\char`\#}}% 
            \def\PYGZpc{\discretionary{}{\Wrappedafterbreak\char`\%}{\char`\%}}% 
            \def\PYGZdl{\discretionary{}{\Wrappedafterbreak\char`\$}{\char`\$}}% 
            \def\PYGZhy{\discretionary{\char`\-}{\Wrappedafterbreak}{\char`\-}}% 
            \def\PYGZsq{\discretionary{}{\Wrappedafterbreak\textquotesingle}{\textquotesingle}}% 
            \def\PYGZdq{\discretionary{}{\Wrappedafterbreak\char`\"}{\char`\"}}% 
            \def\PYGZti{\discretionary{\char`\~}{\Wrappedafterbreak}{\char`\~}}% 
        } 
        % Some characters . , ; ? ! / are not pygmentized. 
        % This macro makes them "active" and they will insert potential linebreaks 
        \newcommand*\Wrappedbreaksatpunct {% 
            \lccode`\~`\.\lowercase{\def~}{\discretionary{\hbox{\char`\.}}{\Wrappedafterbreak}{\hbox{\char`\.}}}% 
            \lccode`\~`\,\lowercase{\def~}{\discretionary{\hbox{\char`\,}}{\Wrappedafterbreak}{\hbox{\char`\,}}}% 
            \lccode`\~`\;\lowercase{\def~}{\discretionary{\hbox{\char`\;}}{\Wrappedafterbreak}{\hbox{\char`\;}}}% 
            \lccode`\~`\:\lowercase{\def~}{\discretionary{\hbox{\char`\:}}{\Wrappedafterbreak}{\hbox{\char`\:}}}% 
            \lccode`\~`\?\lowercase{\def~}{\discretionary{\hbox{\char`\?}}{\Wrappedafterbreak}{\hbox{\char`\?}}}% 
            \lccode`\~`\!\lowercase{\def~}{\discretionary{\hbox{\char`\!}}{\Wrappedafterbreak}{\hbox{\char`\!}}}% 
            \lccode`\~`\/\lowercase{\def~}{\discretionary{\hbox{\char`\/}}{\Wrappedafterbreak}{\hbox{\char`\/}}}% 
            \catcode`\.\active
            \catcode`\,\active 
            \catcode`\;\active
            \catcode`\:\active
            \catcode`\?\active
            \catcode`\!\active
            \catcode`\/\active 
            \lccode`\~`\~ 	
        }
    \makeatother

    \let\OriginalVerbatim=\Verbatim
    \makeatletter
    \renewcommand{\Verbatim}[1][1]{%
        %\parskip\z@skip
        \sbox\Wrappedcontinuationbox {\Wrappedcontinuationsymbol}%
        \sbox\Wrappedvisiblespacebox {\FV@SetupFont\Wrappedvisiblespace}%
        \def\FancyVerbFormatLine ##1{\hsize\linewidth
            \vtop{\raggedright\hyphenpenalty\z@\exhyphenpenalty\z@
                \doublehyphendemerits\z@\finalhyphendemerits\z@
                \strut ##1\strut}%
        }%
        % If the linebreak is at a space, the latter will be displayed as visible
        % space at end of first line, and a continuation symbol starts next line.
        % Stretch/shrink are however usually zero for typewriter font.
        \def\FV@Space {%
            \nobreak\hskip\z@ plus\fontdimen3\font minus\fontdimen4\font
            \discretionary{\copy\Wrappedvisiblespacebox}{\Wrappedafterbreak}
            {\kern\fontdimen2\font}%
        }%
        
        % Allow breaks at special characters using \PYG... macros.
        \Wrappedbreaksatspecials
        % Breaks at punctuation characters . , ; ? ! and / need catcode=\active 	
        \OriginalVerbatim[#1,codes*=\Wrappedbreaksatpunct]%
    }
    \makeatother

    % Exact colors from NB
    \definecolor{incolor}{HTML}{303F9F}
    \definecolor{outcolor}{HTML}{D84315}
    \definecolor{cellborder}{HTML}{CFCFCF}
    \definecolor{cellbackground}{HTML}{F7F7F7}
    
    % prompt
    \makeatletter
    \newcommand{\boxspacing}{\kern\kvtcb@left@rule\kern\kvtcb@boxsep}
    \makeatother
    \newcommand{\prompt}[4]{
        \ttfamily\llap{{\color{#2}[#3]:\hspace{3pt}#4}}\vspace{-\baselineskip}
    }
    

    
    % Prevent overflowing lines due to hard-to-break entities
    \sloppy 
    % Setup hyperref package
    \hypersetup{
      breaklinks=true,  % so long urls are correctly broken across lines
      colorlinks=true,
      urlcolor=urlcolor,
      linkcolor=linkcolor,
      citecolor=citecolor,
      }
    % Slightly bigger margins than the latex defaults
    
    \geometry{verbose,tmargin=1in,bmargin=1in,lmargin=1in,rmargin=1in}
    
    

\begin{document}
    
    \maketitle
    
    

    
    k-means Clustering

    \hypertarget{introduction}{%
\subsection{Introduction}\label{introduction}}

There are many models for clustering out there. In this lab, we will be
presenting the model that is considered the one of the simplest model
among them. Despite its simplicity, \emph{k}-means is vastly used for
clustering in many data science applications, especially useful if you
need to quickly discover insights from unlabeled data.

Some real-world applications of \emph{k}-means include: - customer
segmentation, - understand what the visitors of a website are trying to
accomplish, - pattern recognition, and, - data compression.

In this lab, we will learn \emph{k}-means clustering with 3 examples: -
\emph{k}-means on a randomly generated dataset. - Using \emph{k}-means
for customer segmentation.

    \hypertarget{table-of-contents}{%
\subsection{Table of Contents}\label{table-of-contents}}

\begin{enumerate}
\def\labelenumi{\arabic{enumi}.}
\tightlist
\item
  k-means on a Randomly Generated Dataset\\
\item
  Using k for Customer Segmentation
\end{enumerate}

    Before we start with the main lab content, let's download all the
dependencies that we will need.

    \begin{tcolorbox}[breakable, size=fbox, boxrule=1pt, pad at break*=1mm,colback=cellbackground, colframe=cellborder]
\prompt{In}{incolor}{1}{\boxspacing}
\begin{Verbatim}[commandchars=\\\{\}]
\PY{k+kn}{import} \PY{n+nn}{random} \PY{c+c1}{\PYZsh{} library for random number generation}
\PY{k+kn}{import} \PY{n+nn}{numpy} \PY{k}{as} \PY{n+nn}{np} \PY{c+c1}{\PYZsh{} library for vectorized computation}
\PY{k+kn}{import} \PY{n+nn}{pandas} \PY{k}{as} \PY{n+nn}{pd} \PY{c+c1}{\PYZsh{} library to process data as dataframes}

\PY{k+kn}{import} \PY{n+nn}{matplotlib}\PY{n+nn}{.}\PY{n+nn}{pyplot} \PY{k}{as} \PY{n+nn}{plt} \PY{c+c1}{\PYZsh{} plotting library}
\PY{c+c1}{\PYZsh{} backend for rendering plots within the browser}
\PY{o}{\PYZpc{}}\PY{k}{matplotlib} inline 

\PY{k+kn}{from} \PY{n+nn}{sklearn}\PY{n+nn}{.}\PY{n+nn}{cluster} \PY{k+kn}{import} \PY{n}{KMeans} 
\PY{k+kn}{from} \PY{n+nn}{sklearn}\PY{n+nn}{.}\PY{n+nn}{datasets}\PY{n+nn}{.}\PY{n+nn}{samples\PYZus{}generator} \PY{k+kn}{import} \PY{n}{make\PYZus{}blobs}

\PY{n+nb}{print}\PY{p}{(}\PY{l+s+s1}{\PYZsq{}}\PY{l+s+s1}{Libraries imported.}\PY{l+s+s1}{\PYZsq{}}\PY{p}{)}
\end{Verbatim}
\end{tcolorbox}

    \begin{Verbatim}[commandchars=\\\{\}]
Libraries imported.
    \end{Verbatim}

    

    \hypertarget{k-means-on-a-randomly-generated-dataset}{%
\subsection{\texorpdfstring{1. \emph{k}-means on a Randomly Generated
Dataset}{1. k-means on a Randomly Generated Dataset}}\label{k-means-on-a-randomly-generated-dataset}}

    Let's first demonstrate how \emph{k}-means works with an example of
engineered datapoints.

    \hypertarget{data-points-belonging-to-2-different-clusters-x1-is-the-first-feature-and-x2-is-the-second-feature}{%
\paragraph{30 data points belonging to 2 different clusters (x1 is the
first feature and x2 is the second
feature)}\label{data-points-belonging-to-2-different-clusters-x1-is-the-first-feature-and-x2-is-the-second-feature}}

    \begin{tcolorbox}[breakable, size=fbox, boxrule=1pt, pad at break*=1mm,colback=cellbackground, colframe=cellborder]
\prompt{In}{incolor}{2}{\boxspacing}
\begin{Verbatim}[commandchars=\\\{\}]
\PY{c+c1}{\PYZsh{} data}
\PY{n}{x1} \PY{o}{=} \PY{p}{[}\PY{o}{\PYZhy{}}\PY{l+m+mf}{4.9}\PY{p}{,} \PY{o}{\PYZhy{}}\PY{l+m+mf}{3.5}\PY{p}{,} \PY{l+m+mi}{0}\PY{p}{,} \PY{o}{\PYZhy{}}\PY{l+m+mf}{4.5}\PY{p}{,} \PY{o}{\PYZhy{}}\PY{l+m+mi}{3}\PY{p}{,} \PY{o}{\PYZhy{}}\PY{l+m+mi}{1}\PY{p}{,} \PY{o}{\PYZhy{}}\PY{l+m+mf}{1.2}\PY{p}{,} \PY{o}{\PYZhy{}}\PY{l+m+mf}{4.5}\PY{p}{,} \PY{o}{\PYZhy{}}\PY{l+m+mf}{1.5}\PY{p}{,} \PY{o}{\PYZhy{}}\PY{l+m+mf}{4.5}\PY{p}{,} \PY{o}{\PYZhy{}}\PY{l+m+mi}{1}\PY{p}{,} \PY{o}{\PYZhy{}}\PY{l+m+mi}{2}\PY{p}{,} \PY{o}{\PYZhy{}}\PY{l+m+mf}{2.5}\PY{p}{,} \PY{o}{\PYZhy{}}\PY{l+m+mi}{2}\PY{p}{,} \PY{o}{\PYZhy{}}\PY{l+m+mf}{1.5}\PY{p}{,} \PY{l+m+mi}{4}\PY{p}{,} \PY{l+m+mf}{1.8}\PY{p}{,} \PY{l+m+mi}{2}\PY{p}{,} \PY{l+m+mf}{2.5}\PY{p}{,} \PY{l+m+mi}{3}\PY{p}{,} \PY{l+m+mi}{4}\PY{p}{,} \PY{l+m+mf}{2.25}\PY{p}{,} \PY{l+m+mi}{1}\PY{p}{,} \PY{l+m+mi}{0}\PY{p}{,} \PY{l+m+mi}{1}\PY{p}{,} \PY{l+m+mf}{2.5}\PY{p}{,} \PY{l+m+mi}{5}\PY{p}{,} \PY{l+m+mf}{2.8}\PY{p}{,} \PY{l+m+mi}{2}\PY{p}{,} \PY{l+m+mi}{2}\PY{p}{]}
\PY{n}{x2} \PY{o}{=} \PY{p}{[}\PY{o}{\PYZhy{}}\PY{l+m+mf}{3.5}\PY{p}{,} \PY{o}{\PYZhy{}}\PY{l+m+mi}{4}\PY{p}{,} \PY{o}{\PYZhy{}}\PY{l+m+mf}{3.5}\PY{p}{,} \PY{o}{\PYZhy{}}\PY{l+m+mi}{3}\PY{p}{,} \PY{o}{\PYZhy{}}\PY{l+m+mf}{2.9}\PY{p}{,} \PY{o}{\PYZhy{}}\PY{l+m+mi}{3}\PY{p}{,} \PY{o}{\PYZhy{}}\PY{l+m+mf}{2.6}\PY{p}{,} \PY{o}{\PYZhy{}}\PY{l+m+mf}{2.1}\PY{p}{,} \PY{l+m+mi}{0}\PY{p}{,} \PY{o}{\PYZhy{}}\PY{l+m+mf}{0.5}\PY{p}{,} \PY{o}{\PYZhy{}}\PY{l+m+mf}{0.8}\PY{p}{,} \PY{o}{\PYZhy{}}\PY{l+m+mf}{0.8}\PY{p}{,} \PY{o}{\PYZhy{}}\PY{l+m+mf}{1.5}\PY{p}{,} \PY{o}{\PYZhy{}}\PY{l+m+mf}{1.75}\PY{p}{,} \PY{o}{\PYZhy{}}\PY{l+m+mf}{1.75}\PY{p}{,} \PY{l+m+mi}{0}\PY{p}{,} \PY{l+m+mf}{0.8}\PY{p}{,} \PY{l+m+mf}{0.9}\PY{p}{,} \PY{l+m+mi}{1}\PY{p}{,} \PY{l+m+mi}{1}\PY{p}{,} \PY{l+m+mi}{1}\PY{p}{,} \PY{l+m+mf}{1.75}\PY{p}{,} \PY{l+m+mi}{2}\PY{p}{,} \PY{l+m+mf}{2.5}\PY{p}{,} \PY{l+m+mf}{2.5}\PY{p}{,} \PY{l+m+mf}{2.5}\PY{p}{,} \PY{l+m+mf}{2.5}\PY{p}{,} \PY{l+m+mi}{3}\PY{p}{,} \PY{l+m+mi}{6}\PY{p}{,} \PY{l+m+mf}{6.5}\PY{p}{]}

\PY{n+nb}{print}\PY{p}{(}\PY{l+s+s1}{\PYZsq{}}\PY{l+s+s1}{Datapoints defined!}\PY{l+s+s1}{\PYZsq{}}\PY{p}{)}
\end{Verbatim}
\end{tcolorbox}

    \begin{Verbatim}[commandchars=\\\{\}]
Datapoints defined!
    \end{Verbatim}

    \hypertarget{define-a-function-that-assigns-each-datapoint-to-a-cluster}{%
\paragraph{Define a function that assigns each datapoint to a
cluster}\label{define-a-function-that-assigns-each-datapoint-to-a-cluster}}

    \begin{tcolorbox}[breakable, size=fbox, boxrule=1pt, pad at break*=1mm,colback=cellbackground, colframe=cellborder]
\prompt{In}{incolor}{3}{\boxspacing}
\begin{Verbatim}[commandchars=\\\{\}]
\PY{n}{colors\PYZus{}map} \PY{o}{=} \PY{n}{np}\PY{o}{.}\PY{n}{array}\PY{p}{(}\PY{p}{[}\PY{l+s+s1}{\PYZsq{}}\PY{l+s+s1}{b}\PY{l+s+s1}{\PYZsq{}}\PY{p}{,} \PY{l+s+s1}{\PYZsq{}}\PY{l+s+s1}{r}\PY{l+s+s1}{\PYZsq{}}\PY{p}{]}\PY{p}{)}
\PY{k}{def} \PY{n+nf}{assign\PYZus{}members}\PY{p}{(}\PY{n}{x1}\PY{p}{,} \PY{n}{x2}\PY{p}{,} \PY{n}{centers}\PY{p}{)}\PY{p}{:}
    \PY{n}{compare\PYZus{}to\PYZus{}first\PYZus{}center} \PY{o}{=} \PY{n}{np}\PY{o}{.}\PY{n}{sqrt}\PY{p}{(}\PY{n}{np}\PY{o}{.}\PY{n}{square}\PY{p}{(}\PY{n}{np}\PY{o}{.}\PY{n}{array}\PY{p}{(}\PY{n}{x1}\PY{p}{)} \PY{o}{\PYZhy{}} \PY{n}{centers}\PY{p}{[}\PY{l+m+mi}{0}\PY{p}{]}\PY{p}{[}\PY{l+m+mi}{0}\PY{p}{]}\PY{p}{)} \PY{o}{+} \PY{n}{np}\PY{o}{.}\PY{n}{square}\PY{p}{(}\PY{n}{np}\PY{o}{.}\PY{n}{array}\PY{p}{(}\PY{n}{x2}\PY{p}{)} \PY{o}{\PYZhy{}} \PY{n}{centers}\PY{p}{[}\PY{l+m+mi}{0}\PY{p}{]}\PY{p}{[}\PY{l+m+mi}{1}\PY{p}{]}\PY{p}{)}\PY{p}{)}
    \PY{n}{compare\PYZus{}to\PYZus{}second\PYZus{}center} \PY{o}{=} \PY{n}{np}\PY{o}{.}\PY{n}{sqrt}\PY{p}{(}\PY{n}{np}\PY{o}{.}\PY{n}{square}\PY{p}{(}\PY{n}{np}\PY{o}{.}\PY{n}{array}\PY{p}{(}\PY{n}{x1}\PY{p}{)} \PY{o}{\PYZhy{}} \PY{n}{centers}\PY{p}{[}\PY{l+m+mi}{1}\PY{p}{]}\PY{p}{[}\PY{l+m+mi}{0}\PY{p}{]}\PY{p}{)} \PY{o}{+} \PY{n}{np}\PY{o}{.}\PY{n}{square}\PY{p}{(}\PY{n}{np}\PY{o}{.}\PY{n}{array}\PY{p}{(}\PY{n}{x2}\PY{p}{)} \PY{o}{\PYZhy{}} \PY{n}{centers}\PY{p}{[}\PY{l+m+mi}{1}\PY{p}{]}\PY{p}{[}\PY{l+m+mi}{1}\PY{p}{]}\PY{p}{)}\PY{p}{)}
    \PY{n}{class\PYZus{}of\PYZus{}points} \PY{o}{=} \PY{n}{compare\PYZus{}to\PYZus{}first\PYZus{}center} \PY{o}{\PYZgt{}} \PY{n}{compare\PYZus{}to\PYZus{}second\PYZus{}center}
    \PY{n}{colors} \PY{o}{=} \PY{n}{colors\PYZus{}map}\PY{p}{[}\PY{n}{class\PYZus{}of\PYZus{}points} \PY{o}{+} \PY{l+m+mi}{1} \PY{o}{\PYZhy{}} \PY{l+m+mi}{1}\PY{p}{]}
    \PY{k}{return} \PY{n}{colors}\PY{p}{,} \PY{n}{class\PYZus{}of\PYZus{}points}

\PY{n+nb}{print}\PY{p}{(}\PY{l+s+s1}{\PYZsq{}}\PY{l+s+s1}{assign\PYZus{}members function defined!}\PY{l+s+s1}{\PYZsq{}}\PY{p}{)}
\end{Verbatim}
\end{tcolorbox}

    \begin{Verbatim}[commandchars=\\\{\}]
assign\_members function defined!
    \end{Verbatim}

    \hypertarget{define-a-function-that-updates-the-centroid-of-each-cluster}{%
\paragraph{Define a function that updates the centroid of each
cluster}\label{define-a-function-that-updates-the-centroid-of-each-cluster}}

    \begin{tcolorbox}[breakable, size=fbox, boxrule=1pt, pad at break*=1mm,colback=cellbackground, colframe=cellborder]
\prompt{In}{incolor}{4}{\boxspacing}
\begin{Verbatim}[commandchars=\\\{\}]
\PY{c+c1}{\PYZsh{} update means}
\PY{k}{def} \PY{n+nf}{update\PYZus{}centers}\PY{p}{(}\PY{n}{x1}\PY{p}{,} \PY{n}{x2}\PY{p}{,} \PY{n}{class\PYZus{}of\PYZus{}points}\PY{p}{)}\PY{p}{:}
    \PY{n}{center1} \PY{o}{=} \PY{p}{[}\PY{n}{np}\PY{o}{.}\PY{n}{mean}\PY{p}{(}\PY{n}{np}\PY{o}{.}\PY{n}{array}\PY{p}{(}\PY{n}{x1}\PY{p}{)}\PY{p}{[}\PY{o}{\PYZti{}}\PY{n}{class\PYZus{}of\PYZus{}points}\PY{p}{]}\PY{p}{)}\PY{p}{,} \PY{n}{np}\PY{o}{.}\PY{n}{mean}\PY{p}{(}\PY{n}{np}\PY{o}{.}\PY{n}{array}\PY{p}{(}\PY{n}{x2}\PY{p}{)}\PY{p}{[}\PY{o}{\PYZti{}}\PY{n}{class\PYZus{}of\PYZus{}points}\PY{p}{]}\PY{p}{)}\PY{p}{]}
    \PY{n}{center2} \PY{o}{=} \PY{p}{[}\PY{n}{np}\PY{o}{.}\PY{n}{mean}\PY{p}{(}\PY{n}{np}\PY{o}{.}\PY{n}{array}\PY{p}{(}\PY{n}{x1}\PY{p}{)}\PY{p}{[}\PY{n}{class\PYZus{}of\PYZus{}points}\PY{p}{]}\PY{p}{)}\PY{p}{,} \PY{n}{np}\PY{o}{.}\PY{n}{mean}\PY{p}{(}\PY{n}{np}\PY{o}{.}\PY{n}{array}\PY{p}{(}\PY{n}{x2}\PY{p}{)}\PY{p}{[}\PY{n}{class\PYZus{}of\PYZus{}points}\PY{p}{]}\PY{p}{)}\PY{p}{]}
    \PY{k}{return} \PY{p}{[}\PY{n}{center1}\PY{p}{,} \PY{n}{center2}\PY{p}{]}

\PY{n+nb}{print}\PY{p}{(}\PY{l+s+s1}{\PYZsq{}}\PY{l+s+s1}{assign\PYZus{}members function defined!}\PY{l+s+s1}{\PYZsq{}}\PY{p}{)}
\end{Verbatim}
\end{tcolorbox}

    \begin{Verbatim}[commandchars=\\\{\}]
assign\_members function defined!
    \end{Verbatim}

    \hypertarget{define-a-function-that-plots-the-data-points-along-with-the-cluster-centroids}{%
\paragraph{Define a function that plots the data points along with the
cluster
centroids}\label{define-a-function-that-plots-the-data-points-along-with-the-cluster-centroids}}

    \begin{tcolorbox}[breakable, size=fbox, boxrule=1pt, pad at break*=1mm,colback=cellbackground, colframe=cellborder]
\prompt{In}{incolor}{5}{\boxspacing}
\begin{Verbatim}[commandchars=\\\{\}]
\PY{k}{def} \PY{n+nf}{plot\PYZus{}points}\PY{p}{(}\PY{n}{centroids}\PY{o}{=}\PY{k+kc}{None}\PY{p}{,} \PY{n}{colors}\PY{o}{=}\PY{l+s+s1}{\PYZsq{}}\PY{l+s+s1}{g}\PY{l+s+s1}{\PYZsq{}}\PY{p}{,} \PY{n}{figure\PYZus{}title}\PY{o}{=}\PY{k+kc}{None}\PY{p}{)}\PY{p}{:}
    \PY{c+c1}{\PYZsh{} plot the figure}
    \PY{n}{fig} \PY{o}{=} \PY{n}{plt}\PY{o}{.}\PY{n}{figure}\PY{p}{(}\PY{n}{figsize}\PY{o}{=}\PY{p}{(}\PY{l+m+mi}{15}\PY{p}{,} \PY{l+m+mi}{10}\PY{p}{)}\PY{p}{)}  \PY{c+c1}{\PYZsh{} create a figure object}
    \PY{n}{ax} \PY{o}{=} \PY{n}{fig}\PY{o}{.}\PY{n}{add\PYZus{}subplot}\PY{p}{(}\PY{l+m+mi}{1}\PY{p}{,} \PY{l+m+mi}{1}\PY{p}{,} \PY{l+m+mi}{1}\PY{p}{)}
    
    \PY{n}{centroid\PYZus{}colors} \PY{o}{=} \PY{p}{[}\PY{l+s+s1}{\PYZsq{}}\PY{l+s+s1}{bx}\PY{l+s+s1}{\PYZsq{}}\PY{p}{,} \PY{l+s+s1}{\PYZsq{}}\PY{l+s+s1}{rx}\PY{l+s+s1}{\PYZsq{}}\PY{p}{]}
    \PY{k}{if} \PY{n}{centroids}\PY{p}{:}
        \PY{k}{for} \PY{p}{(}\PY{n}{i}\PY{p}{,} \PY{n}{centroid}\PY{p}{)} \PY{o+ow}{in} \PY{n+nb}{enumerate}\PY{p}{(}\PY{n}{centroids}\PY{p}{)}\PY{p}{:}
            \PY{n}{ax}\PY{o}{.}\PY{n}{plot}\PY{p}{(}\PY{n}{centroid}\PY{p}{[}\PY{l+m+mi}{0}\PY{p}{]}\PY{p}{,} \PY{n}{centroid}\PY{p}{[}\PY{l+m+mi}{1}\PY{p}{]}\PY{p}{,} \PY{n}{centroid\PYZus{}colors}\PY{p}{[}\PY{n}{i}\PY{p}{]}\PY{p}{,} \PY{n}{markeredgewidth}\PY{o}{=}\PY{l+m+mi}{5}\PY{p}{,} \PY{n}{markersize}\PY{o}{=}\PY{l+m+mi}{20}\PY{p}{)}
    \PY{n}{plt}\PY{o}{.}\PY{n}{scatter}\PY{p}{(}\PY{n}{x1}\PY{p}{,} \PY{n}{x2}\PY{p}{,} \PY{n}{s}\PY{o}{=}\PY{l+m+mi}{500}\PY{p}{,} \PY{n}{c}\PY{o}{=}\PY{n}{colors}\PY{p}{)}
    
    \PY{c+c1}{\PYZsh{} define the ticks}
    \PY{n}{xticks} \PY{o}{=} \PY{n}{np}\PY{o}{.}\PY{n}{linspace}\PY{p}{(}\PY{o}{\PYZhy{}}\PY{l+m+mi}{6}\PY{p}{,} \PY{l+m+mi}{8}\PY{p}{,} \PY{l+m+mi}{15}\PY{p}{,} \PY{n}{endpoint}\PY{o}{=}\PY{k+kc}{True}\PY{p}{)}
    \PY{n}{yticks} \PY{o}{=} \PY{n}{np}\PY{o}{.}\PY{n}{linspace}\PY{p}{(}\PY{o}{\PYZhy{}}\PY{l+m+mi}{6}\PY{p}{,} \PY{l+m+mi}{6}\PY{p}{,} \PY{l+m+mi}{13}\PY{p}{,} \PY{n}{endpoint}\PY{o}{=}\PY{k+kc}{True}\PY{p}{)}

    \PY{c+c1}{\PYZsh{} fix the horizontal axis}
    \PY{n}{ax}\PY{o}{.}\PY{n}{set\PYZus{}xticks}\PY{p}{(}\PY{n}{xticks}\PY{p}{)}
    \PY{n}{ax}\PY{o}{.}\PY{n}{set\PYZus{}yticks}\PY{p}{(}\PY{n}{yticks}\PY{p}{)}

    \PY{c+c1}{\PYZsh{} add tick labels}
    \PY{n}{xlabels} \PY{o}{=} \PY{n}{xticks}
    \PY{n}{ax}\PY{o}{.}\PY{n}{set\PYZus{}xticklabels}\PY{p}{(}\PY{n}{xlabels}\PY{p}{)}
    \PY{n}{ylabels} \PY{o}{=} \PY{n}{yticks}
    \PY{n}{ax}\PY{o}{.}\PY{n}{set\PYZus{}yticklabels}\PY{p}{(}\PY{n}{ylabels}\PY{p}{)}

    \PY{c+c1}{\PYZsh{} style the ticks}
    \PY{n}{ax}\PY{o}{.}\PY{n}{xaxis}\PY{o}{.}\PY{n}{set\PYZus{}ticks\PYZus{}position}\PY{p}{(}\PY{l+s+s1}{\PYZsq{}}\PY{l+s+s1}{bottom}\PY{l+s+s1}{\PYZsq{}}\PY{p}{)}
    \PY{n}{ax}\PY{o}{.}\PY{n}{yaxis}\PY{o}{.}\PY{n}{set\PYZus{}ticks\PYZus{}position}\PY{p}{(}\PY{l+s+s1}{\PYZsq{}}\PY{l+s+s1}{left}\PY{l+s+s1}{\PYZsq{}}\PY{p}{)}
    \PY{n}{ax}\PY{o}{.}\PY{n}{tick\PYZus{}params}\PY{p}{(}\PY{l+s+s1}{\PYZsq{}}\PY{l+s+s1}{both}\PY{l+s+s1}{\PYZsq{}}\PY{p}{,} \PY{n}{length}\PY{o}{=}\PY{l+m+mi}{2}\PY{p}{,} \PY{n}{width}\PY{o}{=}\PY{l+m+mi}{1}\PY{p}{,} \PY{n}{which}\PY{o}{=}\PY{l+s+s1}{\PYZsq{}}\PY{l+s+s1}{major}\PY{l+s+s1}{\PYZsq{}}\PY{p}{,} \PY{n}{labelsize}\PY{o}{=}\PY{l+m+mi}{15}\PY{p}{)}
    
    \PY{c+c1}{\PYZsh{} add labels to axes}
    \PY{n}{ax}\PY{o}{.}\PY{n}{set\PYZus{}xlabel}\PY{p}{(}\PY{l+s+s1}{\PYZsq{}}\PY{l+s+s1}{x1}\PY{l+s+s1}{\PYZsq{}}\PY{p}{,} \PY{n}{fontsize}\PY{o}{=}\PY{l+m+mi}{20}\PY{p}{)}
    \PY{n}{ax}\PY{o}{.}\PY{n}{set\PYZus{}ylabel}\PY{p}{(}\PY{l+s+s1}{\PYZsq{}}\PY{l+s+s1}{x2}\PY{l+s+s1}{\PYZsq{}}\PY{p}{,} \PY{n}{fontsize}\PY{o}{=}\PY{l+m+mi}{20}\PY{p}{)}
    
    \PY{c+c1}{\PYZsh{} add title to figure}
    \PY{n}{ax}\PY{o}{.}\PY{n}{set\PYZus{}title}\PY{p}{(}\PY{n}{figure\PYZus{}title}\PY{p}{,} \PY{n}{fontsize}\PY{o}{=}\PY{l+m+mi}{24}\PY{p}{)}

    \PY{n}{plt}\PY{o}{.}\PY{n}{show}\PY{p}{(}\PY{p}{)}

\PY{n+nb}{print}\PY{p}{(}\PY{l+s+s1}{\PYZsq{}}\PY{l+s+s1}{plot\PYZus{}points function defined!}\PY{l+s+s1}{\PYZsq{}}\PY{p}{)}
\end{Verbatim}
\end{tcolorbox}

    \begin{Verbatim}[commandchars=\\\{\}]
plot\_points function defined!
    \end{Verbatim}

    \hypertarget{initialize-k-means---plot-data-points}{%
\paragraph{\texorpdfstring{Initialize \emph{k}-means - plot data
points}{Initialize k-means - plot data points}}\label{initialize-k-means---plot-data-points}}

    \begin{tcolorbox}[breakable, size=fbox, boxrule=1pt, pad at break*=1mm,colback=cellbackground, colframe=cellborder]
\prompt{In}{incolor}{6}{\boxspacing}
\begin{Verbatim}[commandchars=\\\{\}]
\PY{n}{plot\PYZus{}points}\PY{p}{(}\PY{n}{figure\PYZus{}title}\PY{o}{=}\PY{l+s+s1}{\PYZsq{}}\PY{l+s+s1}{Scatter Plot of x2 vs x1}\PY{l+s+s1}{\PYZsq{}}\PY{p}{)}
\end{Verbatim}
\end{tcolorbox}

    \begin{center}
    \adjustimage{max size={0.9\linewidth}{0.9\paperheight}}{output_17_0.png}
    \end{center}
    { \hspace*{\fill} \\}
    
    \hypertarget{initialize-k-means---randomly-define-clusters-and-add-them-to-plot}{%
\paragraph{\texorpdfstring{Initialize \emph{k}-means - randomly define
clusters and add them to
plot}{Initialize k-means - randomly define clusters and add them to plot}}\label{initialize-k-means---randomly-define-clusters-and-add-them-to-plot}}

    \begin{tcolorbox}[breakable, size=fbox, boxrule=1pt, pad at break*=1mm,colback=cellbackground, colframe=cellborder]
\prompt{In}{incolor}{7}{\boxspacing}
\begin{Verbatim}[commandchars=\\\{\}]
\PY{n}{centers} \PY{o}{=} \PY{p}{[}\PY{p}{[}\PY{o}{\PYZhy{}}\PY{l+m+mi}{2}\PY{p}{,} \PY{l+m+mi}{2}\PY{p}{]}\PY{p}{,} \PY{p}{[}\PY{l+m+mi}{2}\PY{p}{,} \PY{o}{\PYZhy{}}\PY{l+m+mi}{2}\PY{p}{]}\PY{p}{]}
\PY{n}{plot\PYZus{}points}\PY{p}{(}\PY{n}{centers}\PY{p}{,} \PY{n}{figure\PYZus{}title}\PY{o}{=}\PY{l+s+s1}{\PYZsq{}}\PY{l+s+s1}{k\PYZhy{}means Initialization}\PY{l+s+s1}{\PYZsq{}}\PY{p}{)}
\end{Verbatim}
\end{tcolorbox}

    \begin{center}
    \adjustimage{max size={0.9\linewidth}{0.9\paperheight}}{output_19_0.png}
    \end{center}
    { \hspace*{\fill} \\}
    
    \hypertarget{run-k-means-4-iterations-only}{%
\paragraph{\texorpdfstring{Run \emph{k}-means (4-iterations
only)}{Run k-means (4-iterations only)}}\label{run-k-means-4-iterations-only}}

    \begin{tcolorbox}[breakable, size=fbox, boxrule=1pt, pad at break*=1mm,colback=cellbackground, colframe=cellborder]
\prompt{In}{incolor}{8}{\boxspacing}
\begin{Verbatim}[commandchars=\\\{\}]
\PY{n}{number\PYZus{}of\PYZus{}iterations} \PY{o}{=} \PY{l+m+mi}{4}
\PY{k}{for} \PY{n}{i} \PY{o+ow}{in} \PY{n+nb}{range}\PY{p}{(}\PY{n}{number\PYZus{}of\PYZus{}iterations}\PY{p}{)}\PY{p}{:}
    \PY{n+nb}{input}\PY{p}{(}\PY{l+s+s1}{\PYZsq{}}\PY{l+s+s1}{Iteration }\PY{l+s+si}{\PYZob{}\PYZcb{}}\PY{l+s+s1}{ \PYZhy{} Press Enter to update the members of each cluster}\PY{l+s+s1}{\PYZsq{}}\PY{o}{.}\PY{n}{format}\PY{p}{(}\PY{n}{i} \PY{o}{+} \PY{l+m+mi}{1}\PY{p}{)}\PY{p}{)}
    \PY{n}{colors}\PY{p}{,} \PY{n}{class\PYZus{}of\PYZus{}points} \PY{o}{=} \PY{n}{assign\PYZus{}members}\PY{p}{(}\PY{n}{x1}\PY{p}{,} \PY{n}{x2}\PY{p}{,} \PY{n}{centers}\PY{p}{)}
    \PY{n}{title} \PY{o}{=} \PY{l+s+s1}{\PYZsq{}}\PY{l+s+s1}{Iteration }\PY{l+s+si}{\PYZob{}\PYZcb{}}\PY{l+s+s1}{ \PYZhy{} Cluster Assignment}\PY{l+s+s1}{\PYZsq{}}\PY{o}{.}\PY{n}{format}\PY{p}{(}\PY{n}{i} \PY{o}{+} \PY{l+m+mi}{1}\PY{p}{)}
    \PY{n}{plot\PYZus{}points}\PY{p}{(}\PY{n}{centers}\PY{p}{,} \PY{n}{colors}\PY{p}{,} \PY{n}{figure\PYZus{}title}\PY{o}{=}\PY{n}{title}\PY{p}{)}
    \PY{n+nb}{input}\PY{p}{(}\PY{l+s+s1}{\PYZsq{}}\PY{l+s+s1}{Iteration }\PY{l+s+si}{\PYZob{}\PYZcb{}}\PY{l+s+s1}{ \PYZhy{} Press Enter to update the centers}\PY{l+s+s1}{\PYZsq{}}\PY{o}{.}\PY{n}{format}\PY{p}{(}\PY{n}{i} \PY{o}{+} \PY{l+m+mi}{1}\PY{p}{)}\PY{p}{)}
    \PY{n}{centers} \PY{o}{=} \PY{n}{update\PYZus{}centers}\PY{p}{(}\PY{n}{x1}\PY{p}{,} \PY{n}{x2}\PY{p}{,} \PY{n}{class\PYZus{}of\PYZus{}points}\PY{p}{)}
    \PY{n}{title} \PY{o}{=} \PY{l+s+s1}{\PYZsq{}}\PY{l+s+s1}{Iteration }\PY{l+s+si}{\PYZob{}\PYZcb{}}\PY{l+s+s1}{ \PYZhy{} Centroid Update}\PY{l+s+s1}{\PYZsq{}}\PY{o}{.}\PY{n}{format}\PY{p}{(}\PY{n}{i} \PY{o}{+} \PY{l+m+mi}{1}\PY{p}{)}
    \PY{n}{plot\PYZus{}points}\PY{p}{(}\PY{n}{centers}\PY{p}{,} \PY{n}{colors}\PY{p}{,} \PY{n}{figure\PYZus{}title}\PY{o}{=}\PY{n}{title}\PY{p}{)}
\end{Verbatim}
\end{tcolorbox}

    \begin{Verbatim}[commandchars=\\\{\}]
Iteration 1 - Press Enter to update the members of each cluster a
    \end{Verbatim}

    \begin{center}
    \adjustimage{max size={0.9\linewidth}{0.9\paperheight}}{output_21_1.png}
    \end{center}
    { \hspace*{\fill} \\}
    
    \begin{Verbatim}[commandchars=\\\{\}]
Iteration 1 - Press Enter to update the centers b
    \end{Verbatim}

    \begin{center}
    \adjustimage{max size={0.9\linewidth}{0.9\paperheight}}{output_21_3.png}
    \end{center}
    { \hspace*{\fill} \\}
    
    \begin{Verbatim}[commandchars=\\\{\}]
Iteration 2 - Press Enter to update the members of each cluster c
    \end{Verbatim}

    \begin{center}
    \adjustimage{max size={0.9\linewidth}{0.9\paperheight}}{output_21_5.png}
    \end{center}
    { \hspace*{\fill} \\}
    
    \begin{Verbatim}[commandchars=\\\{\}]
Iteration 2 - Press Enter to update the centers d
    \end{Verbatim}

    \begin{center}
    \adjustimage{max size={0.9\linewidth}{0.9\paperheight}}{output_21_7.png}
    \end{center}
    { \hspace*{\fill} \\}
    
    \begin{Verbatim}[commandchars=\\\{\}]
Iteration 3 - Press Enter to update the members of each cluster
    \end{Verbatim}

    \begin{center}
    \adjustimage{max size={0.9\linewidth}{0.9\paperheight}}{output_21_9.png}
    \end{center}
    { \hspace*{\fill} \\}
    
    \begin{Verbatim}[commandchars=\\\{\}]
Iteration 3 - Press Enter to update the centers
    \end{Verbatim}

    \begin{center}
    \adjustimage{max size={0.9\linewidth}{0.9\paperheight}}{output_21_11.png}
    \end{center}
    { \hspace*{\fill} \\}
    
    \begin{Verbatim}[commandchars=\\\{\}]
Iteration 4 - Press Enter to update the members of each cluster
    \end{Verbatim}

    \begin{center}
    \adjustimage{max size={0.9\linewidth}{0.9\paperheight}}{output_21_13.png}
    \end{center}
    { \hspace*{\fill} \\}
    
    \begin{Verbatim}[commandchars=\\\{\}]
Iteration 4 - Press Enter to update the centers
    \end{Verbatim}

    \begin{center}
    \adjustimage{max size={0.9\linewidth}{0.9\paperheight}}{output_21_15.png}
    \end{center}
    { \hspace*{\fill} \\}
    
    Now, we have visually observed how k-means works, let's look at an
example with many more datapoints. For this example, we will use the
random library to generate thousands of datapoints.

    \hypertarget{generating-the-data}{%
\subsubsection{Generating the Data}\label{generating-the-data}}

    First, we need to set up a random seed. We use the Numpy's
\textbf{random.seed()} function, and we will set the seed to 0. In other
words, \textbf{random.seed(0)}.

    \begin{tcolorbox}[breakable, size=fbox, boxrule=1pt, pad at break*=1mm,colback=cellbackground, colframe=cellborder]
\prompt{In}{incolor}{9}{\boxspacing}
\begin{Verbatim}[commandchars=\\\{\}]
\PY{n}{np}\PY{o}{.}\PY{n}{random}\PY{o}{.}\PY{n}{seed}\PY{p}{(}\PY{l+m+mi}{0}\PY{p}{)}
\end{Verbatim}
\end{tcolorbox}

    Next we will be making \emph{random clusters} of points by using the
\textbf{make\_blobs} class. The \textbf{make\_blobs} class can take in
many inputs, but we will use these specific ones.

 Input

n\_samples: The total number of points equally divided among clusters.

Value will be: 5000

centers: The number of centers to generate, or the fixed center
locations.

Value will be: {[}{[}4, 4{]}, {[}-2, -1{]}, {[}2, -3{]},{[}1,1{]}{]}

cluster\_std: The standard deviation of the clusters.

Value will be: 0.9

 Output

X: Array of shape {[}n\_samples, n\_features{]}. (Feature Matrix)

The generated samples.

y: Array of shape {[}n\_samples{]}. (Response Vector)

The integer labels for cluster membership of each sample.

    \begin{tcolorbox}[breakable, size=fbox, boxrule=1pt, pad at break*=1mm,colback=cellbackground, colframe=cellborder]
\prompt{In}{incolor}{10}{\boxspacing}
\begin{Verbatim}[commandchars=\\\{\}]
\PY{n}{X}\PY{p}{,} \PY{n}{y} \PY{o}{=} \PY{n}{make\PYZus{}blobs}\PY{p}{(}\PY{n}{n\PYZus{}samples}\PY{o}{=}\PY{l+m+mi}{5000}\PY{p}{,} \PY{n}{centers}\PY{o}{=}\PY{p}{[}\PY{p}{[}\PY{l+m+mi}{4}\PY{p}{,} \PY{l+m+mi}{4}\PY{p}{]}\PY{p}{,} \PY{p}{[}\PY{o}{\PYZhy{}}\PY{l+m+mi}{2}\PY{p}{,} \PY{o}{\PYZhy{}}\PY{l+m+mi}{1}\PY{p}{]}\PY{p}{,} \PY{p}{[}\PY{l+m+mi}{2}\PY{p}{,} \PY{o}{\PYZhy{}}\PY{l+m+mi}{3}\PY{p}{]}\PY{p}{,} \PY{p}{[}\PY{l+m+mi}{1}\PY{p}{,} \PY{l+m+mi}{1}\PY{p}{]}\PY{p}{]}\PY{p}{,} \PY{n}{cluster\PYZus{}std}\PY{o}{=}\PY{l+m+mf}{0.9}\PY{p}{)}
\end{Verbatim}
\end{tcolorbox}

    Display the scatter plot of the randomly generated data.

    \begin{tcolorbox}[breakable, size=fbox, boxrule=1pt, pad at break*=1mm,colback=cellbackground, colframe=cellborder]
\prompt{In}{incolor}{11}{\boxspacing}
\begin{Verbatim}[commandchars=\\\{\}]
\PY{n}{plt}\PY{o}{.}\PY{n}{figure}\PY{p}{(}\PY{n}{figsize}\PY{o}{=}\PY{p}{(}\PY{l+m+mi}{15}\PY{p}{,} \PY{l+m+mi}{10}\PY{p}{)}\PY{p}{)}
\PY{n}{plt}\PY{o}{.}\PY{n}{scatter}\PY{p}{(}\PY{n}{X}\PY{p}{[}\PY{p}{:}\PY{p}{,} \PY{l+m+mi}{0}\PY{p}{]}\PY{p}{,} \PY{n}{X}\PY{p}{[}\PY{p}{:}\PY{p}{,} \PY{l+m+mi}{1}\PY{p}{]}\PY{p}{,} \PY{n}{marker}\PY{o}{=}\PY{l+s+s1}{\PYZsq{}}\PY{l+s+s1}{.}\PY{l+s+s1}{\PYZsq{}}\PY{p}{)}
\end{Verbatim}
\end{tcolorbox}

            \begin{tcolorbox}[breakable, size=fbox, boxrule=.5pt, pad at break*=1mm, opacityfill=0]
\prompt{Out}{outcolor}{11}{\boxspacing}
\begin{Verbatim}[commandchars=\\\{\}]
<matplotlib.collections.PathCollection at 0x7fc14d858080>
\end{Verbatim}
\end{tcolorbox}
        
    \begin{center}
    \adjustimage{max size={0.9\linewidth}{0.9\paperheight}}{output_29_1.png}
    \end{center}
    { \hspace*{\fill} \\}
    
    \hypertarget{setting-up-k-means}{%
\subsubsection{\texorpdfstring{Setting up
\emph{k}-means}{Setting up k-means}}\label{setting-up-k-means}}

    Now that we have our random data, let's set up our \emph{k}-means
clustering.

    The KMeans class has many parameters that can be used, but we will use
these three:

init: Initialization method of the centroids.

Value will be: ``k-means++''. k-means++ selects initial cluster centers
for k-means clustering in a smart way to speed up convergence.

n\_clusters: The number of clusters to form as well as the number of
centroids to generate.

Value will be: 4 (since we have 4 centers)

n\_init: Number of times the k-means algorithm will be run with
different centroid seeds. The final results will be the best output of
n\_init consecutive runs in terms of inertia.

Value will be: 12

Initialize KMeans with these parameters, where the output parameter is
called \textbf{k\_means}.

    \begin{tcolorbox}[breakable, size=fbox, boxrule=1pt, pad at break*=1mm,colback=cellbackground, colframe=cellborder]
\prompt{In}{incolor}{12}{\boxspacing}
\begin{Verbatim}[commandchars=\\\{\}]
\PY{n}{k\PYZus{}means} \PY{o}{=} \PY{n}{KMeans}\PY{p}{(}\PY{n}{init}\PY{o}{=}\PY{l+s+s2}{\PYZdq{}}\PY{l+s+s2}{k\PYZhy{}means++}\PY{l+s+s2}{\PYZdq{}}\PY{p}{,} \PY{n}{n\PYZus{}clusters}\PY{o}{=}\PY{l+m+mi}{4}\PY{p}{,} \PY{n}{n\PYZus{}init}\PY{o}{=}\PY{l+m+mi}{12}\PY{p}{)}
\end{Verbatim}
\end{tcolorbox}

    Now let's fit the KMeans model with the feature matrix we created above,
X .

    \begin{tcolorbox}[breakable, size=fbox, boxrule=1pt, pad at break*=1mm,colback=cellbackground, colframe=cellborder]
\prompt{In}{incolor}{13}{\boxspacing}
\begin{Verbatim}[commandchars=\\\{\}]
\PY{n}{k\PYZus{}means}\PY{o}{.}\PY{n}{fit}\PY{p}{(}\PY{n}{X}\PY{p}{)}
\end{Verbatim}
\end{tcolorbox}

            \begin{tcolorbox}[breakable, size=fbox, boxrule=.5pt, pad at break*=1mm, opacityfill=0]
\prompt{Out}{outcolor}{13}{\boxspacing}
\begin{Verbatim}[commandchars=\\\{\}]
KMeans(algorithm='auto', copy\_x=True, init='k-means++', max\_iter=300,
    n\_clusters=4, n\_init=12, n\_jobs=None, precompute\_distances='auto',
    random\_state=None, tol=0.0001, verbose=0)
\end{Verbatim}
\end{tcolorbox}
        
    Now let's grab the labels for each point in the model using KMeans
\textbf{.labels\_} attribute and save it as \textbf{k\_means\_labels}.

    \begin{tcolorbox}[breakable, size=fbox, boxrule=1pt, pad at break*=1mm,colback=cellbackground, colframe=cellborder]
\prompt{In}{incolor}{14}{\boxspacing}
\begin{Verbatim}[commandchars=\\\{\}]
\PY{n}{k\PYZus{}means\PYZus{}labels} \PY{o}{=} \PY{n}{k\PYZus{}means}\PY{o}{.}\PY{n}{labels\PYZus{}}
\PY{n}{k\PYZus{}means\PYZus{}labels}
\end{Verbatim}
\end{tcolorbox}

            \begin{tcolorbox}[breakable, size=fbox, boxrule=.5pt, pad at break*=1mm, opacityfill=0]
\prompt{Out}{outcolor}{14}{\boxspacing}
\begin{Verbatim}[commandchars=\\\{\}]
array([0, 3, 3, {\ldots}, 1, 0, 0], dtype=int32)
\end{Verbatim}
\end{tcolorbox}
        
    We will also get the coordinates of the cluster centers using KMeans
\textbf{.cluster\_centers\_} and save it as
\textbf{k\_means\_cluster\_centers}.

    \begin{tcolorbox}[breakable, size=fbox, boxrule=1pt, pad at break*=1mm,colback=cellbackground, colframe=cellborder]
\prompt{In}{incolor}{15}{\boxspacing}
\begin{Verbatim}[commandchars=\\\{\}]
\PY{n}{k\PYZus{}means\PYZus{}cluster\PYZus{}centers} \PY{o}{=} \PY{n}{k\PYZus{}means}\PY{o}{.}\PY{n}{cluster\PYZus{}centers\PYZus{}}
\PY{n}{k\PYZus{}means\PYZus{}cluster\PYZus{}centers}
\end{Verbatim}
\end{tcolorbox}

            \begin{tcolorbox}[breakable, size=fbox, boxrule=.5pt, pad at break*=1mm, opacityfill=0]
\prompt{Out}{outcolor}{15}{\boxspacing}
\begin{Verbatim}[commandchars=\\\{\}]
array([[-2.03743147, -0.99782524],
       [ 3.97334234,  3.98758687],
       [ 0.96900523,  0.98370298],
       [ 1.99741008, -3.01666822]])
\end{Verbatim}
\end{tcolorbox}
        
    \hypertarget{visualizing-the-resulting-clusters}{%
\subsubsection{Visualizing the Resulting
Clusters}\label{visualizing-the-resulting-clusters}}

    So now that we have the random data generated and the KMeans model
initialized, let's plot them and see what the clusters look like.

    Please read through the code and comments to understand how to plot the
model.

    \begin{tcolorbox}[breakable, size=fbox, boxrule=1pt, pad at break*=1mm,colback=cellbackground, colframe=cellborder]
\prompt{In}{incolor}{16}{\boxspacing}
\begin{Verbatim}[commandchars=\\\{\}]
\PY{c+c1}{\PYZsh{} initialize the plot with the specified dimensions.}
\PY{n}{fig} \PY{o}{=} \PY{n}{plt}\PY{o}{.}\PY{n}{figure}\PY{p}{(}\PY{n}{figsize}\PY{o}{=}\PY{p}{(}\PY{l+m+mi}{15}\PY{p}{,} \PY{l+m+mi}{10}\PY{p}{)}\PY{p}{)}

\PY{c+c1}{\PYZsh{} colors uses a color map, which will produce an array of colors based on}
\PY{c+c1}{\PYZsh{} the number of labels. We use set(k\PYZus{}means\PYZus{}labels) to get the}
\PY{c+c1}{\PYZsh{} unique labels.}
\PY{n}{colors} \PY{o}{=} \PY{n}{plt}\PY{o}{.}\PY{n}{cm}\PY{o}{.}\PY{n}{Spectral}\PY{p}{(}\PY{n}{np}\PY{o}{.}\PY{n}{linspace}\PY{p}{(}\PY{l+m+mi}{0}\PY{p}{,} \PY{l+m+mi}{1}\PY{p}{,} \PY{n+nb}{len}\PY{p}{(}\PY{n+nb}{set}\PY{p}{(}\PY{n}{k\PYZus{}means\PYZus{}labels}\PY{p}{)}\PY{p}{)}\PY{p}{)}\PY{p}{)}

\PY{c+c1}{\PYZsh{} create a plot}
\PY{n}{ax} \PY{o}{=} \PY{n}{fig}\PY{o}{.}\PY{n}{add\PYZus{}subplot}\PY{p}{(}\PY{l+m+mi}{1}\PY{p}{,} \PY{l+m+mi}{1}\PY{p}{,} \PY{l+m+mi}{1}\PY{p}{)}

\PY{c+c1}{\PYZsh{} loop through the data and plot the datapoints and centroids.}
\PY{c+c1}{\PYZsh{} k will range from 0\PYZhy{}3, which will match the number of clusters in the dataset.}
\PY{k}{for} \PY{n}{k}\PY{p}{,} \PY{n}{col} \PY{o+ow}{in} \PY{n+nb}{zip}\PY{p}{(}\PY{n+nb}{range}\PY{p}{(}\PY{n+nb}{len}\PY{p}{(}\PY{p}{[}\PY{p}{[}\PY{l+m+mi}{4}\PY{p}{,}\PY{l+m+mi}{4}\PY{p}{]}\PY{p}{,} \PY{p}{[}\PY{o}{\PYZhy{}}\PY{l+m+mi}{2}\PY{p}{,} \PY{o}{\PYZhy{}}\PY{l+m+mi}{1}\PY{p}{]}\PY{p}{,} \PY{p}{[}\PY{l+m+mi}{2}\PY{p}{,} \PY{o}{\PYZhy{}}\PY{l+m+mi}{3}\PY{p}{]}\PY{p}{,} \PY{p}{[}\PY{l+m+mi}{1}\PY{p}{,} \PY{l+m+mi}{1}\PY{p}{]}\PY{p}{]}\PY{p}{)}\PY{p}{)}\PY{p}{,} \PY{n}{colors}\PY{p}{)}\PY{p}{:}

    \PY{c+c1}{\PYZsh{} create a list of all datapoints, where the datapoitns that are }
    \PY{c+c1}{\PYZsh{} in the cluster (ex. cluster 0) are labeled as true, else they are}
    \PY{c+c1}{\PYZsh{} labeled as false.}
    \PY{n}{my\PYZus{}members} \PY{o}{=} \PY{p}{(}\PY{n}{k\PYZus{}means\PYZus{}labels} \PY{o}{==} \PY{n}{k}\PY{p}{)}
    
    \PY{c+c1}{\PYZsh{} define the centroid, or cluster center.}
    \PY{n}{cluster\PYZus{}center} \PY{o}{=} \PY{n}{k\PYZus{}means\PYZus{}cluster\PYZus{}centers}\PY{p}{[}\PY{n}{k}\PY{p}{]}
    
    \PY{c+c1}{\PYZsh{} plot the datapoints with color col.}
    \PY{n}{ax}\PY{o}{.}\PY{n}{plot}\PY{p}{(}\PY{n}{X}\PY{p}{[}\PY{n}{my\PYZus{}members}\PY{p}{,} \PY{l+m+mi}{0}\PY{p}{]}\PY{p}{,} \PY{n}{X}\PY{p}{[}\PY{n}{my\PYZus{}members}\PY{p}{,} \PY{l+m+mi}{1}\PY{p}{]}\PY{p}{,} \PY{l+s+s1}{\PYZsq{}}\PY{l+s+s1}{w}\PY{l+s+s1}{\PYZsq{}}\PY{p}{,} \PY{n}{markerfacecolor}\PY{o}{=}\PY{n}{col}\PY{p}{,} \PY{n}{marker}\PY{o}{=}\PY{l+s+s1}{\PYZsq{}}\PY{l+s+s1}{.}\PY{l+s+s1}{\PYZsq{}}\PY{p}{)}
    
    \PY{c+c1}{\PYZsh{} plot the centroids with specified color, but with a darker outline}
    \PY{n}{ax}\PY{o}{.}\PY{n}{plot}\PY{p}{(}\PY{n}{cluster\PYZus{}center}\PY{p}{[}\PY{l+m+mi}{0}\PY{p}{]}\PY{p}{,} \PY{n}{cluster\PYZus{}center}\PY{p}{[}\PY{l+m+mi}{1}\PY{p}{]}\PY{p}{,} \PY{l+s+s1}{\PYZsq{}}\PY{l+s+s1}{o}\PY{l+s+s1}{\PYZsq{}}\PY{p}{,} \PY{n}{markerfacecolor}\PY{o}{=}\PY{n}{col}\PY{p}{,}  \PY{n}{markeredgecolor}\PY{o}{=}\PY{l+s+s1}{\PYZsq{}}\PY{l+s+s1}{k}\PY{l+s+s1}{\PYZsq{}}\PY{p}{,} \PY{n}{markersize}\PY{o}{=}\PY{l+m+mi}{6}\PY{p}{)}

\PY{c+c1}{\PYZsh{} title of the plot}
\PY{n}{ax}\PY{o}{.}\PY{n}{set\PYZus{}title}\PY{p}{(}\PY{l+s+s1}{\PYZsq{}}\PY{l+s+s1}{KMeans}\PY{l+s+s1}{\PYZsq{}}\PY{p}{)}

\PY{c+c1}{\PYZsh{} remove x\PYZhy{}axis ticks}
\PY{n}{ax}\PY{o}{.}\PY{n}{set\PYZus{}xticks}\PY{p}{(}\PY{p}{(}\PY{p}{)}\PY{p}{)}

\PY{c+c1}{\PYZsh{} remove y\PYZhy{}axis ticks}
\PY{n}{ax}\PY{o}{.}\PY{n}{set\PYZus{}yticks}\PY{p}{(}\PY{p}{(}\PY{p}{)}\PY{p}{)}

\PY{c+c1}{\PYZsh{} show the plot}
\PY{n}{plt}\PY{o}{.}\PY{n}{show}\PY{p}{(}\PY{p}{)}
\end{Verbatim}
\end{tcolorbox}

    \begin{center}
    \adjustimage{max size={0.9\linewidth}{0.9\paperheight}}{output_43_0.png}
    \end{center}
    { \hspace*{\fill} \\}
    
    

    \hypertarget{using-k-means-for-customer-segmentation}{%
\subsection{\texorpdfstring{2. Using \emph{k}-means for Customer
Segmentation}{2. Using k-means for Customer Segmentation}}\label{using-k-means-for-customer-segmentation}}

    Imagine that you have a customer dataset, and you are interested in
exploring the behavior of your customers using their historical data.

Customer segmentation~is the practice of partitioning a customer base
into groups of individuals that have similar characteristics. It is a
significant strategy as a business can target these specific groups of
customers and effectively allocate marketing resources. For example, one
group might contain customers who are high-profit and low-risk, that is,
more likely to purchase products, or subscribe to a service. A business
task is to retain those customers. Another group might include customers
from non-profit organizations, and so on.

    \hypertarget{downloading-data}{%
\subsubsection{Downloading Data}\label{downloading-data}}

    Let's download the data and save it as a CSV file called
\textbf{customer\_segmentation.csv}

    \begin{tcolorbox}[breakable, size=fbox, boxrule=1pt, pad at break*=1mm,colback=cellbackground, colframe=cellborder]
\prompt{In}{incolor}{17}{\boxspacing}
\begin{Verbatim}[commandchars=\\\{\}]
\PY{o}{!}wget \PYZhy{}q \PYZhy{}O \PY{l+s+s1}{\PYZsq{}customer\PYZus{}segmentation.csv\PYZsq{}} https://cocl.us/customer\PYZus{}dataset
\PY{n+nb}{print}\PY{p}{(}\PY{l+s+s1}{\PYZsq{}}\PY{l+s+s1}{Data downloaded!}\PY{l+s+s1}{\PYZsq{}}\PY{p}{)}
\end{Verbatim}
\end{tcolorbox}

    \begin{Verbatim}[commandchars=\\\{\}]
Data downloaded!
    \end{Verbatim}

    Now that the data is downloaded, let's read it into a \emph{pandas}
dataframe.

    \begin{tcolorbox}[breakable, size=fbox, boxrule=1pt, pad at break*=1mm,colback=cellbackground, colframe=cellborder]
\prompt{In}{incolor}{18}{\boxspacing}
\begin{Verbatim}[commandchars=\\\{\}]
\PY{n}{customers\PYZus{}df} \PY{o}{=} \PY{n}{pd}\PY{o}{.}\PY{n}{read\PYZus{}csv}\PY{p}{(}\PY{l+s+s1}{\PYZsq{}}\PY{l+s+s1}{customer\PYZus{}segmentation.csv}\PY{l+s+s1}{\PYZsq{}}\PY{p}{)}
\PY{n}{customers\PYZus{}df}\PY{o}{.}\PY{n}{head}\PY{p}{(}\PY{p}{)}
\end{Verbatim}
\end{tcolorbox}

            \begin{tcolorbox}[breakable, size=fbox, boxrule=.5pt, pad at break*=1mm, opacityfill=0]
\prompt{Out}{outcolor}{18}{\boxspacing}
\begin{Verbatim}[commandchars=\\\{\}]
   Customer Id  Age  Edu  Years Employed  Income  Card Debt  Other Debt  \textbackslash{}
0            1   41    2               6      19      0.124       1.073
1            2   47    1              26     100      4.582       8.218
2            3   33    2              10      57      6.111       5.802
3            4   29    2               4      19      0.681       0.516
4            5   47    1              31     253      9.308       8.908

   Defaulted Address  DebtIncomeRatio
0        0.0  NBA001              6.3
1        0.0  NBA021             12.8
2        1.0  NBA013             20.9
3        0.0  NBA009              6.3
4        0.0  NBA008              7.2
\end{Verbatim}
\end{tcolorbox}
        
    \hypertarget{pre-processing}{%
\subsubsection{Pre-processing}\label{pre-processing}}

    As you can see, \textbf{Address} in this dataset is a categorical
variable. k-means algorithm isn't directly applicable to categorical
variables because Euclidean distance function isn't really meaningful
for discrete variables. So, lets drop this feature and run clustering.

    \begin{tcolorbox}[breakable, size=fbox, boxrule=1pt, pad at break*=1mm,colback=cellbackground, colframe=cellborder]
\prompt{In}{incolor}{19}{\boxspacing}
\begin{Verbatim}[commandchars=\\\{\}]
\PY{n}{df} \PY{o}{=} \PY{n}{customers\PYZus{}df}\PY{o}{.}\PY{n}{drop}\PY{p}{(}\PY{l+s+s1}{\PYZsq{}}\PY{l+s+s1}{Address}\PY{l+s+s1}{\PYZsq{}}\PY{p}{,} \PY{n}{axis}\PY{o}{=}\PY{l+m+mi}{1}\PY{p}{)}
\PY{n}{df}\PY{o}{.}\PY{n}{head}\PY{p}{(}\PY{p}{)}
\end{Verbatim}
\end{tcolorbox}

            \begin{tcolorbox}[breakable, size=fbox, boxrule=.5pt, pad at break*=1mm, opacityfill=0]
\prompt{Out}{outcolor}{19}{\boxspacing}
\begin{Verbatim}[commandchars=\\\{\}]
   Customer Id  Age  Edu  Years Employed  Income  Card Debt  Other Debt  \textbackslash{}
0            1   41    2               6      19      0.124       1.073
1            2   47    1              26     100      4.582       8.218
2            3   33    2              10      57      6.111       5.802
3            4   29    2               4      19      0.681       0.516
4            5   47    1              31     253      9.308       8.908

   Defaulted  DebtIncomeRatio
0        0.0              6.3
1        0.0             12.8
2        1.0             20.9
3        0.0              6.3
4        0.0              7.2
\end{Verbatim}
\end{tcolorbox}
        
    Now let's normalize the dataset. But why do we need normalization in the
first place? Normalization is a statistical method that helps
mathematical-based algorithms interpret features with different
magnitudes and distributions equally. We use \textbf{StandardScaler()}
to normalize our dataset.

    \begin{tcolorbox}[breakable, size=fbox, boxrule=1pt, pad at break*=1mm,colback=cellbackground, colframe=cellborder]
\prompt{In}{incolor}{20}{\boxspacing}
\begin{Verbatim}[commandchars=\\\{\}]
\PY{k+kn}{from} \PY{n+nn}{sklearn}\PY{n+nn}{.}\PY{n+nn}{preprocessing} \PY{k+kn}{import} \PY{n}{StandardScaler}

\PY{n}{X} \PY{o}{=} \PY{n}{df}\PY{o}{.}\PY{n}{values}\PY{p}{[}\PY{p}{:}\PY{p}{,}\PY{l+m+mi}{1}\PY{p}{:}\PY{p}{]}
\PY{n}{X} \PY{o}{=} \PY{n}{np}\PY{o}{.}\PY{n}{nan\PYZus{}to\PYZus{}num}\PY{p}{(}\PY{n}{X}\PY{p}{)}
\PY{n}{cluster\PYZus{}dataset} \PY{o}{=} \PY{n}{StandardScaler}\PY{p}{(}\PY{p}{)}\PY{o}{.}\PY{n}{fit\PYZus{}transform}\PY{p}{(}\PY{n}{X}\PY{p}{)}
\PY{n}{cluster\PYZus{}dataset}
\end{Verbatim}
\end{tcolorbox}

            \begin{tcolorbox}[breakable, size=fbox, boxrule=.5pt, pad at break*=1mm, opacityfill=0]
\prompt{Out}{outcolor}{20}{\boxspacing}
\begin{Verbatim}[commandchars=\\\{\}]
array([[ 0.74291541,  0.31212243, -0.37878978, {\ldots}, -0.59048916,
        -0.52379654, -0.57652509],
       [ 1.48949049, -0.76634938,  2.5737211 , {\ldots},  1.51296181,
        -0.52379654,  0.39138677],
       [-0.25251804,  0.31212243,  0.2117124 , {\ldots},  0.80170393,
         1.90913822,  1.59755385],
       {\ldots},
       [-1.24795149,  2.46906604, -1.26454304, {\ldots},  0.03863257,
         1.90913822,  3.45892281],
       [-0.37694723, -0.76634938,  0.50696349, {\ldots}, -0.70147601,
        -0.52379654, -1.08281745],
       [ 2.1116364 , -0.76634938,  1.09746566, {\ldots},  0.16463355,
        -0.52379654, -0.2340332 ]])
\end{Verbatim}
\end{tcolorbox}
        
    \hypertarget{modeling}{%
\subsubsection{Modeling}\label{modeling}}

    Let's run our model and group our customers into three clusters.

    \begin{tcolorbox}[breakable, size=fbox, boxrule=1pt, pad at break*=1mm,colback=cellbackground, colframe=cellborder]
\prompt{In}{incolor}{21}{\boxspacing}
\begin{Verbatim}[commandchars=\\\{\}]
\PY{n}{num\PYZus{}clusters} \PY{o}{=} \PY{l+m+mi}{3}

\PY{n}{k\PYZus{}means} \PY{o}{=} \PY{n}{KMeans}\PY{p}{(}\PY{n}{init}\PY{o}{=}\PY{l+s+s2}{\PYZdq{}}\PY{l+s+s2}{k\PYZhy{}means++}\PY{l+s+s2}{\PYZdq{}}\PY{p}{,} \PY{n}{n\PYZus{}clusters}\PY{o}{=}\PY{n}{num\PYZus{}clusters}\PY{p}{,} \PY{n}{n\PYZus{}init}\PY{o}{=}\PY{l+m+mi}{12}\PY{p}{)}
\PY{n}{k\PYZus{}means}\PY{o}{.}\PY{n}{fit}\PY{p}{(}\PY{n}{cluster\PYZus{}dataset}\PY{p}{)}
\PY{n}{labels} \PY{o}{=} \PY{n}{k\PYZus{}means}\PY{o}{.}\PY{n}{labels\PYZus{}}

\PY{n+nb}{print}\PY{p}{(}\PY{n}{labels}\PY{p}{)}
\end{Verbatim}
\end{tcolorbox}

    \begin{Verbatim}[commandchars=\\\{\}]
[1 2 0 1 2 2 1 1 1 2 0 1 1 1 0 1 1 1 2 1 1 1 0 2 2 1 1 1 1 1 1 2 0 1 1 1 0
 0 1 2 0 2 1 2 1 2 1 1 1 1 2 2 0 1 0 0 0 1 1 1 2 1 2 2 1 1 1 0 1 0 1 1 1 1
 1 1 1 1 2 1 1 0 2 1 2 1 1 1 0 0 1 1 0 0 1 1 1 1 0 1 0 2 1 0 0 2 1 1 1 1 1
 1 1 0 1 1 1 1 2 1 1 1 1 1 1 1 1 1 1 1 1 1 1 1 1 1 1 1 1 0 0 1 1 1 2 1 1 0
 1 1 1 2 0 1 1 1 1 2 0 0 1 2 1 1 1 1 1 1 1 1 2 1 1 0 1 0 1 1 0 2 0 1 1 2 0
 2 1 1 1 1 1 2 1 0 1 1 1 2 2 1 2 1 0 1 1 0 1 2 1 0 1 1 1 1 1 0 0 2 1 1 0 2
 1 1 1 1 2 1 1 0 1 1 1 1 2 1 1 0 2 1 1 1 1 1 1 2 1 2 1 1 1 1 1 1 2 0 1 0 1
 1 1 2 1 0 2 0 1 2 1 1 0 1 1 1 1 0 0 0 1 1 1 2 1 1 2 1 2 1 1 2 1 1 1 0 1 1
 0 1 0 2 1 1 1 1 0 1 1 0 0 1 1 1 1 1 1 1 1 0 1 0 2 1 0 1 1 1 0 0 1 1 1 2 0
 1 1 0 1 2 1 1 1 1 1 0 2 1 1 1 1 1 2 1 1 1 1 1 2 1 1 1 2 0 1 0 1 1 1 2 2 1
 0 1 2 0 0 1 1 1 0 1 1 1 1 1 2 1 2 1 1 1 1 0 1 0 1 1 1 2 1 1 1 1 0 1 1 0 0
 2 1 1 1 1 1 0 0 1 2 0 2 1 1 0 1 1 2 2 1 0 1 1 2 1 0 1 2 1 1 1 2 1 1 1 1 2
 1 0 1 1 1 1 2 0 1 1 2 1 0 1 1 2 1 2 1 1 1 1 1 1 1 2 2 1 1 2 1 0 1 1 1 0 1
 0 1 1 1 1 1 2 0 0 1 2 1 2 1 1 0 2 1 0 0 0 2 2 0 1 1 0 1 0 0 1 0 2 1 1 0 1
 0 2 0 1 1 0 1 1 0 0 0 1 1 1 2 2 1 1 0 1 1 0 2 1 0 1 1 1 0 1 2 1 2 2 1 2 1
 1 2 1 0 1 1 1 1 0 0 1 2 1 2 1 1 2 1 0 1 0 1 0 0 0 2 0 1 1 1 0 1 1 1 2 1 2
 1 0 0 1 1 1 1 1 1 1 0 2 1 2 1 1 0 1 1 1 0 1 1 0 0 0 0 1 2 1 0 0 1 1 1 1 2
 2 1 0 1 1 2 1 1 2 1 2 1 1 2 0 2 2 2 0 1 1 0 1 2 2 1 1 1 2 0 1 1 1 1 2 1 1
 1 1 1 0 1 1 2 1 1 2 1 1 1 1 1 1 0 2 1 1 0 1 1 1 1 0 1 2 1 1 2 1 1 0 1 0 1
 0 0 1 1 1 2 0 2 1 2 2 1 0 1 2 1 2 1 1 1 1 1 2 1 0 1 1 2 2 1 1 2 1 1 1 1 1
 1 1 1 0 1 1 2 1 1 1 1 1 1 1 0 1 1 1 2 0 2 2 1 1 1 0 1 1 1 0 0 1 0 1 1 1 2
 1 1 1 1 1 1 1 2 1 1 1 1 1 1 2 2 0 0 1 0 1 1 1 1 2 0 1 1 1 1 1 2 0 1 1 1 0
 1 1 0 1 1 1 1 1 1 0 0 2 2 1 1 1 0 1 1 1 1 1 1 1 1 1 1 1 1 1 1 1 1 0 1 2]
    \end{Verbatim}

    Insights

    Note that each row in our dataset represents a customer, and therefore,
each row is assigned a label.

    \begin{tcolorbox}[breakable, size=fbox, boxrule=1pt, pad at break*=1mm,colback=cellbackground, colframe=cellborder]
\prompt{In}{incolor}{22}{\boxspacing}
\begin{Verbatim}[commandchars=\\\{\}]
\PY{n}{df}\PY{p}{[}\PY{l+s+s2}{\PYZdq{}}\PY{l+s+s2}{Labels}\PY{l+s+s2}{\PYZdq{}}\PY{p}{]} \PY{o}{=} \PY{n}{labels}
\PY{n}{df}\PY{o}{.}\PY{n}{head}\PY{p}{(}\PY{l+m+mi}{5}\PY{p}{)}
\end{Verbatim}
\end{tcolorbox}

            \begin{tcolorbox}[breakable, size=fbox, boxrule=.5pt, pad at break*=1mm, opacityfill=0]
\prompt{Out}{outcolor}{22}{\boxspacing}
\begin{Verbatim}[commandchars=\\\{\}]
   Customer Id  Age  Edu  Years Employed  Income  Card Debt  Other Debt  \textbackslash{}
0            1   41    2               6      19      0.124       1.073
1            2   47    1              26     100      4.582       8.218
2            3   33    2              10      57      6.111       5.802
3            4   29    2               4      19      0.681       0.516
4            5   47    1              31     253      9.308       8.908

   Defaulted  DebtIncomeRatio  Labels
0        0.0              6.3       1
1        0.0             12.8       2
2        1.0             20.9       0
3        0.0              6.3       1
4        0.0              7.2       2
\end{Verbatim}
\end{tcolorbox}
        
    We can easily check the centroid values by averaging the features in
each cluster.

    \begin{tcolorbox}[breakable, size=fbox, boxrule=1pt, pad at break*=1mm,colback=cellbackground, colframe=cellborder]
\prompt{In}{incolor}{23}{\boxspacing}
\begin{Verbatim}[commandchars=\\\{\}]
\PY{n}{df}\PY{o}{.}\PY{n}{groupby}\PY{p}{(}\PY{l+s+s1}{\PYZsq{}}\PY{l+s+s1}{Labels}\PY{l+s+s1}{\PYZsq{}}\PY{p}{)}\PY{o}{.}\PY{n}{mean}\PY{p}{(}\PY{p}{)}
\end{Verbatim}
\end{tcolorbox}

            \begin{tcolorbox}[breakable, size=fbox, boxrule=.5pt, pad at break*=1mm, opacityfill=0]
\prompt{Out}{outcolor}{23}{\boxspacing}
\begin{Verbatim}[commandchars=\\\{\}]
        Customer Id        Age       Edu  Years Employed      Income  \textbackslash{}
Labels
0        424.451807  31.891566  1.861446        3.963855   31.789157
1        426.122905  33.817505  1.603352        7.625698   36.143389
2        424.408163  43.000000  1.931973       17.197279  101.959184

        Card Debt  Other Debt  Defaulted  DebtIncomeRatio
Labels
0        1.576675    2.843355   0.993939        13.994578
1        0.853128    1.816855   0.000000         7.964991
2        4.220673    7.954483   0.162393        13.915646
\end{Verbatim}
\end{tcolorbox}
        
    k-means will partition your customers into three groups since we
specified the algorithm to generate 3 clusters. The customers in each
cluster are similar to each other in terms of the features included in
the dataset.

Now we can create a profile for each group, considering the common
characteristics of each cluster. For example, the 3 clusters can be:

\begin{itemize}
\tightlist
\item
  OLDER, HIGH INCOME, AND INDEBTED
\item
  MIDDLE AGED, MIDDLE INCOME, AND FINANCIALLY RESPONSIBLE
\item
  YOUNG, LOW INCOME, AND INDEBTED
\end{itemize}

    However, you can devise your own profiles based on the means above and
come up with labels that you think best describe each cluster.

    I hope that you are able to see the power of \emph{k}-means here. This
clustering algorithm provided us with insight into the dataset and lead
us to group the data into three clusters. Perhaps the same results would
have been achieved but using multiple tests and experiments.

    \hypertarget{thank-you-for-completing-this-lab}{%
\subsubsection{Thank you for completing this
lab!}\label{thank-you-for-completing-this-lab}}

This notebook was created by
\href{https://ca.linkedin.com/in/saeedaghabozorgi}{Saeed Aghabozorgi}
and \href{https://www.linkedin.com/in/aklson/}{Alex Aklson}. We hope you
found this lab interesting and educational. Feel free to contact us if
you have any questions!

    This notebook is part of a course on \textbf{Coursera} called
\emph{Applied Data Science Capstone}. If you accessed this notebook
outside the course, you can take this course online by clicking
\href{http://cocl.us/DP0701EN_Coursera_Week3_LAB1}{here}.

    Copyright © 2018
\href{https://cognitiveclass.ai/?utm_source=bducopyrightlink\&utm_medium=dswb\&utm_campaign=bdu}{Cognitive
Class}. This notebook and its source code are released under the terms
of the \href{https://bigdatauniversity.com/mit-license/}{MIT License}.


    % Add a bibliography block to the postdoc
    
    
    
\end{document}
